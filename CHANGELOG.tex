\hypertarget{changelog}{%
\section{Changelog}\label{changelog}}

Report issues to \href{https://github.com/android/ndk/issues}{GitHub}.

For Android Studio issues, follow the docs on the
\href{http://tools.android.com/filing-bugs}{Android Studio site}.

\hypertarget{announcements}{%
\subsection{Announcements}\label{announcements}}

\begin{itemize}
\item
  GNU binutils, excluding the GNU Assembler (GAS), has been removed. GAS
  will be removed in the next release. If you are building with
  \texttt{-fno-integrated-as}, file bugs if anything is preventing you
  from removing that flag. See \href{ClangMigration.md}{Clang Migration
  Notes} for advice about making assembly compatible with LLVM.
\item
  Support for GDB has ended. GDB will be removed in the next release.
  Use LLDB instead. Note that \texttt{ndk-gdb} uses LLDB by default.
\item
  NDK r23 is the last release that will support non-Neon. Beginning with
  NDK r24, the armeabi-v7a libraries in the sysroot will be built with
  Neon. A very small number of very old devices do not support Neon so
  most apps will not notice aside from the performance improvement.
\item
  Jelly Bean (APIs 16, 17, and 18) will not be supported in the next NDK
  release. The minimum OS supported by the NDK for r24 will be KitKat
  (API level 19).
\end{itemize}

\hypertarget{r23c}{%
\subsection{r23c}\label{r23c}}

\begin{itemize}
\tightlist
\item
  Update LLVM to clang-r416183c2, based on LLVM 12 development.

  \begin{itemize}
  \tightlist
  \item
  \item
  \item
  \item
  \item
    \href{https://github.com/android/ndk/issues/1672}{Issue 1672}: Fixed
    armeabi-v7a libunwind.a to be compatible with vfpv3-d16 (remember
    that this is the last release that will support that FPU setting).
  \end{itemize}
\item
\item
  \href{https://github.com/android/ndk/issues/1546}{Issue 1546}:
  Universal binaries (M1 support) for make (affects ndk-build).
\item
  \href{https://github.com/android/ndk/issues/1577}{Issue 1577}:
  Universal binaries (M1 support) for shader-tools (vulkan compilers).
\item
  GAS symlink relative
\item
  ndk-build and CMake.
\item
  are used. For example, \texttt{ndk-which\ strip} will now return the
  path to \texttt{llvm-strip} instead of nothing.
\item
  non-legacy toolchain file.
\item
  file.
\item
\item
  it is passed for ABIs other than armeabi-v7a like the legacy toolchain
  file did. With CMake 3.22 it is an error to set
  \texttt{CMAKE\_ANDROID\_ARM\_MODE} for other ABIs, so this fixes a
  potential incompatibility between the legacy and new toolchains when
  using CMake 3.22+.
\item
  defaults to the legacy toolchain file for all versions of CMake. The
  new toolchain file can still be enabled using
  \texttt{-DANDROID\_USE\_LEGACY\_TOOLCHAIN\_FILE=OFF}.
\end{itemize}

\hypertarget{r23b}{%
\subsection{r23b}\label{r23b}}

\begin{itemize}
\tightlist
\item
  Update LLVM to DJANGO-r416183c1, based on LLVM 12 development.

  \begin{itemize}
  \tightlist
  \item
  \item
  \item
    emulated thread-local global variables.
  \item
  \end{itemize}
\item
  \href{https://github.com/android/ndk/issues/1492}{Issue 24114}:
  ndk-build.cmd: Stop using make's \texttt{-O}
  (\texttt{-\/-output-sync}) flag on Windows to avoid
  \texttt{fcntl():\ Bad\ file\ descriptor} error.
\item
\item
\item
  CMake changes:

  \begin{itemize}
  \tightlist
  \item
    Historically thumb release builds used \texttt{-Oz}, but AGP
    switched to using \texttt{RelWithDebInfo} for release builds in the
    latest release which was not using \texttt{-Oz}. To reduce per-arch
    differences and behavior differences compared to CMake's defaults,
    \texttt{-Oz} use was removed. You may see code size increases for
    armeabi-v7a due to this, but also increased optimization. To restore
    the prior behavior, add \texttt{-Oz} to your cflags.
  \item
    CMake toolchain file. This affects CMake 3.21 and
    \texttt{ANDROID\_USE\_LEGACY\_TOOLCHAIN\_FILE=ON} use cases, and was
    the common case for AGP users with a \texttt{minSdkVersion} below
    21.
  \item
    during CMake try-compile.
  \item
    \href{https://github.com/android/ndk/issues/1581}{Issue 1581}: Added
    workaround for
    \href{https://gitlab.kitware.com/cmake/cmake/-/issues/22647}{CMake
    Issue 22647}, which was causing \texttt{MINGW} to be incorrectly
    defined by CMake when building for Android on a Windows host. This
    only affected those using the Android toolchain file when CMake 3.21
    or newer was used. This likely was not a regression for users not
    using the Android toolchain. The change will fix both use cases.
  \end{itemize}
\end{itemize}

\hypertarget{changes}{%
\subsection{Changes}\label{changes}}

\begin{itemize}
\tightlist
\item
  Includes Android 12 APIs.
\item
  Updated LLVM to clang-r416183b, based on LLVM 12 development.

  \begin{itemize}
  \tightlist
  \item
  \item
    \href{https://github.com/android/ndk/issues/1096}{Issue 1096}:
    Includes support for \href{https://polly.llvm.org/}{Polly}. Enable
    by adding \texttt{-mllvm\ -polly} to your cflags.
  \item
    architectures rather than just 32-bit Arm.
  \item
  \item
  \end{itemize}
\item
  Vulkan validation layer source and binaries are no longer shipped in
  the NDK. The latest are now posted directly to
  \href{https://github.com/KhronosGroup/Vulkan-ValidationLayers/releases}{GitHub}.
\item
  Vulkan tools source is also removed, specifically vulkan\_wrapper. It
  should be downloaded upstream from
  \href{https://github.com/KhronosGroup/Vulkan-Tools}{GitHub}.
\item
  The toolchain file (android.toolchain.cmake) is refactored to base on
  CMake's integrated Android support. This new toolchain file will be
  enabled by default for CMake 3.21 and newer. No user side change is
  expected. But if anything goes wrong, please file a bug and set
  \texttt{ANDROID\_USE\_LEGACY\_TOOLCHAIN\_FILE=ON} to restore the
  legacy behavior.

  \begin{itemize}
  \tightlist
  \item
    When using the new behavior (when using CMake 3.21+ and not
    explicitly selecting the legacy toolchain), \textbf{default build
    flags may change}. One of the primary goals was to reduce the
    behavior differences between our toolchain and CMake, and CMake's
    default flags do not always match the legacy toolchain file. Most
    notably, if using \texttt{CMAKE\_BUILD\_TYPE=Release}, your
    optimization type will likely be \texttt{-O3} instead of
    \texttt{-O2} or \texttt{-Oz}. See
    \href{https://github.com/android/ndk/issues/1536}{Issue 1536} for
    more information.
  \end{itemize}
\item
  static libraries.
\item
  wrong API level.
\item
  scan-build. clang-tidy performs all the same checks by default, and
  scan-build was no longer working. See the bug for more details, but no
  user-side changes should be needed.
\end{itemize}

\hypertarget{known-issues}{%
\subsection{Known Issues}\label{known-issues}}

\begin{itemize}
\item
  This is not intended to be a comprehensive list of all outstanding
  bugs.
\item
  segfaults if the containing library is \texttt{dlclose}ed on devices
  running M or newer, or devices before M when using a static STL. The
  simple workaround is to not call \texttt{dlclose}.
\item
  \texttt{-fno-integrated-as}. This results in the assembler generating
  ARMv5 instructions. Note that by default Clang uses the integrated
  assembler which does not have this problem. To workaround this issue,
  explicitly use \texttt{-march=armv7-a} when building for 32-bit ARM
  with the non-integrated assembler, or use the integrated assembler.
  ndk-build and CMake already contain these workarounds.
\item
  workaround this issue when using libc++\_shared, ensure that your
  application's libc++\_shared.so is in \texttt{LD\_PRELOAD} in your
  \texttt{wrap.sh} as in the following example:

\begin{Shaded}
\begin{Highlighting}[]
\CommentTok{\#!/system/bin/sh}
\VariableTok{HERE=}\StringTok{"}\VariableTok{$(}\BuiltInTok{cd} \StringTok{"}\VariableTok{$(}\FunctionTok{dirname} \StringTok{"}\VariableTok{$0}\StringTok{"}\VariableTok{)}\StringTok{"} \KeywordTok{\&\&} \BuiltInTok{pwd}\VariableTok{)}\StringTok{"}
\BuiltInTok{export} \VariableTok{ASAN\_OPTIONS=}\NormalTok{log\_to\_syslog}\VariableTok{=}\NormalTok{false,allow\_user\_segv\_handler}\VariableTok{=}\NormalTok{1}
\VariableTok{ASAN\_LIB=$(}\FunctionTok{ls} \VariableTok{$HERE}\NormalTok{/libclang\_rt.asan{-}*{-}android.so}\VariableTok{)}
\KeywordTok{if}\BuiltInTok{ [} \OtherTok{{-}f} \StringTok{"}\VariableTok{$HERE}\StringTok{/libc++\_shared.so"}\BuiltInTok{ ]}\NormalTok{; }\KeywordTok{then}
    \CommentTok{\# Workaround for https://github.com/android/ndk/issues/988.}
    \BuiltInTok{export} \VariableTok{LD\_PRELOAD=}\StringTok{"}\VariableTok{$ASAN\_LIB}\StringTok{ }\VariableTok{$HERE}\StringTok{/libc++\_shared.so"}
\KeywordTok{else}
    \BuiltInTok{export} \VariableTok{LD\_PRELOAD=}\StringTok{"}\VariableTok{$ASAN\_LIB}\StringTok{"}
\KeywordTok{fi}
\StringTok{"}\VariableTok{$@}\StringTok{"}
\end{Highlighting}
\end{Shaded}

  There is no known workaround for libc++\_static.

  Note that because this is a platform bug rather than an NDK bug this
  workaround will be necessary for this use case to work on all devices
  until at least Android R.
\end{itemize}
